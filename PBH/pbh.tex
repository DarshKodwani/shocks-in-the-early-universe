\documentclass[aps,showpacs,twocolumn,floats,prd,superscriptaddress,nofootinbib]{revtex4} 
\usepackage{graphicx,amsmath,amssymb,amstext}
\usepackage{amssymb,amsbsy,amsfonts,amsthm,color}

\usepackage{epsfig}
%\usepackage{showkeys}
\usepackage{graphicx}
\usepackage{subfigure}

\graphicspath{{Figures/}}

\begin{document}

\title{Shocks and GW from PBH formation}


\author{Ue-Li Pen}
\email{pen@cita.utoronto.ca}
\affiliation{Canadian Institute of Theoretical Astrophysics, 60 St George St, Toronto, ON M5S 3H8, Canada.}
\affiliation{Canadian Institute for Advanced Research, CIFAR program in Gravitation and Cosmology.}
\affiliation{Perimeter Institute of Theoretical Physics, 31 Caroline Street North, Waterloo, ON N2L 2Y5, Canada.}

\author{Neil Turok}
\email{neil@perimeterinstitute.ca}
\affiliation{Perimeter Institute of Theoretical Physics, 31 Caroline Street North, Waterloo, ON N2L 2Y5, Canada.}

\begin{abstract}

\end{abstract}

\maketitle

\section{Introduction and Summary}

recently, \citet{2016arXiv160300464B} proposed that the recent LIGO detection\cite{2016PhRvL.116f1102A}
could be due to mergers of primordial black holes

\acknowledgments


This work is supported by the Canadian Government through the Canadian
Institute for Advance Research and Industry Canada, and by Province of
Ontario through the Ministry of Research and Innovation.

\appendix

%\bibliographystyle{utcaps}
\bibliography{pbh}


\end{document}
